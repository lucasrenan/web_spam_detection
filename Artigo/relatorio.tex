% !TEX encoding = UTF-8 Unicode
\documentclass[10pt, conference, compsocconf]{IEEEtran}
\usepackage[utf8]{inputenc}
\usepackage[portuges,brazil]{babel}

\begin{document}

\title{Bare Demo of IEEEtran.cls for IEEECS Conferences}

\author{
\IEEEauthorblockN{Alan S. Castro}
\IEEEauthorblockA{Universidade Federal de São Carlos\\
UFSCar\\
Sorocaba, Brasil\\
email@gmail.com}\\
\IEEEauthorblockN{Lucas Renan A. Nunes}
\IEEEauthorblockA{Universidade Federal de São Carlos\\
UFSCar\\
Sorocaba, Brasil\\
email@gmail.com}
\and
\IEEEauthorblockN{Cássia C. Monteiro}
\IEEEauthorblockA{Universidade Federal de São Carlos\\
UFSCar\\
Sorocaba, Brasil\\
kssia.cm@gmail.com}\\
\IEEEauthorblockN{Thamires C. Luz}
\IEEEauthorblockA{Universidade Federal de São Carlos\\
UFSCar\\
Sorocaba, Brasil\\
thamiluz@gmail.com}
}


\maketitle 
\begin{abstract}
The abstract goes here. DO NOT USE SPECIAL CHARACTERS, SYMBOLS, OR MATH IN YOUR TITLE OR ABSTRACT.

\end{abstract}

\begin{IEEEkeywords}
web spam; classificador;

\end{IEEEkeywords}

\section{Introdução} 
Com a internet, é comum usarmos sites de busca para encontrarmos resultados como páginas, arquivos, imagens, músicas ou etc . Os resultados exibidos são elencados, através de search engines que geram um rank com os melhores resultados de acordo com o conteúdo relevante na página. O search engine começou por volta de 1994, com o World Wide Web Worm, e tinha por volta de 110.000 páginas web associadas. Após essse ano, isso foi crescendo cada vez mais \cite{1}.
Com o crescimento do uso de sites de busca, surgiu o web spam que segundo \cite{2} é definido pela ação de alterar páginas da internet com a intenção de enganar os search engines, de maneira que o conteúdo inserido para elencar a página no rank, não tenha relação com a página em questão.
\cite{3}, relata que web spam é um dos maiores problemas dos search engines, porque degrada a qualidade dos resultados oferecidos, onde muitas pessoas ficam frustadas ao acessarem páginas que não tem relação com o termo buscado. Ainda relata que o web spam impacta também na área econômica, pois o retorno financeiro de propagandas são aumentadas de acordo com uma boa colocação no rank de busca.
Com o crescimento desse problema, estudos se tornam cada vez mais necessários para detectar se uma página usa de métodos fraudulentos para se beneficiar na posição do rank. Com o uso de técnicas de aprendizado de máquina, podemos classificar as páginas como sendo legítimas ou spam, nesse trabalho comparamos três métodos para classificação de web spam.
Na sessão \ref{descricao}, é realizada a descrição da base de dados utilizada nesse trabalho, e citados os métodos definidos. A sessão \ref{metodos} contém uma explicação mais detalhada dos métodos utilizados. O modelo de comparação

\section{Descrição do trabalho}\label{descricao} 
Para esse trabalho foi disponibilizado uma base de dados com aproximadamente 4 mil amostras, ou seja páginas, e 137 atributos além do atributo classe. Os atributos são dados contínuos, e  descrevem características para páginas que são ou não são spam. A base está balanceada tendo metade das amostras classificadas como spam, e a outra metade classificada como não spam.
De acordo com isso, é proposto a comparação de três métodos de aprendizado de máquina, sendo definido a utilização de Naive Bayes, Regressão Logística e Redes Neurais Artificiais, descritas com detalhe na próxima seção.
Foi utilizado o software livre Octave para implementação e testes necessários.

\section{Métodos}\label{metodos}
\subsection{Naive Bayes}
De acordo com \cite{4}, o classificador bayeseano é uma técnica probabilística baseada no teorema de Thomas Bayes, denominado naive Bayes, onde é considerado um algoritmo "ingênuo", devido assumir que todos os atributos possuem relações independentes entre si, o algoritmo de naive Bayes, pode ser descrito como o produtório das probabilidades de cada atributo pelas classes existentes, onde a amostra é classificada para determinada classe, caso o valor calculado de produtório seja superior ao das outras classes. Comparativos entre algoritmos demonstram que o naive Bayes, obteve resultados compatíveis com os métodos de árvore de decisão e redes neurais. Devido a sua simplicidade e o alto poder preditivo, é um dos algoritmos mais utilizados \cite{5}.
 
\subsection{Regressão Logística}
A regressão logística, é um classificador que através da função sigmoidal classifica um grupo de amostras. \cite{6}, retrata que o método é robusto, flexível e de fácil utilização, com o objetivo de encontrar o melhor modelo que descreva os atríbutos em uma determina classe.

\subsection{Redes Neurais Artifíciais}
Redes Neurais Artifíciais é um método que pode ser usado como classificador, foi inicialmente desenvolvido para funcionar como o sistema neural humano, desde então uma grande variedade de modelos de redes foram desenvolvidos, são compostas por conexões entre neuronios que em conjunto determinam o comportamento da rede, a escolha do modelo da rede depende do problema a ser resolvido, segundo \cite{7} o mais utilizado é o de multilayer perceptrons.

\section{Metodologia}\label{metodologia}


\section{Conclusion}
The conclusion goes here. this is more of the conclusion


\section{section}
teste thamires

%\bibliography{refs}
\bibliographystyle{IEEEtran}
\bibliography{refs}

\end{document}